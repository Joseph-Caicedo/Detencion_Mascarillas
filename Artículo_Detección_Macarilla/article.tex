\documentclass[journal]{IEEEtran}
\usepackage[utf8]{inputenc}

\usepackage{hyperref}
\hypersetup{
    colorlinks=true,
    linkcolor=black,
    filecolor=black,      
    urlcolor=black,
    citecolor=black,
}
\usepackage{float}
\usepackage{dblfloatfix}
\usepackage{cite}
\usepackage{graphicx}
\usepackage{caption}
\usepackage{textcomp}
\usepackage{amsmath,amssymb,amsfonts}
\usepackage{algorithm}
\usepackage{xcolor}
\hyphenation{op-tical net-works semi-conduc-tor}
\graphicspath{ {images/} }

\begin{document}

% Título del artículo 
\title{Detección de mascarilla y de temperatura con Raspberry Pi}

% Autores
\author{Mariana~Rojas~Romero~y
        Joseph~Caicedo~Sáenz
\thanks{Manuscript received April 3, 2021; revised April 10, 2021.}
\thanks{M. Rojas, Universidad Surcolombiana
Huila, Colombia (email: u2017
2162111@usco.edu.co).}% <-this % stops a space
\thanks{J. B. S. Caicedo, Universidad Surcolombiana, Huila, Colombia 
(email: u20171156180@usco.edu.co).}}

% Encabezado 
\markboth{Detección de mascarilla y de temperatura con Raspberry~Pi, April~2021}%
{Shell \MakeLowercase{\textit{et al.}}: Bare Demo of IEEEtran.cls for IEEE Journals}

% Crear el área del título
\maketitle

% abstract and keywords
\begin{abstract}
The abstract goes here.
\end{abstract}

\begin{IEEEkeywords}
Deep Learning, COVID-19.
\end{IEEEkeywords}

\IEEEpeerreviewmaketitle

% Introducción
\section{Introducción}

\IEEEPARstart{L}{a} rápida propagación de COVID-19 obligó a la Organización Mundial de la Salud a declarar la COVID-19 como una pandemia mundial \cite{OPS2020}. Actualmente la COVID-19 es un problema importante de salud pública y que causa fuertes destrozos en la economía debido a los efectos perjudiciales del virus en la calidad de vida de las personas, lo que contribuye a las infecciones respiratorias agudas, la mortalidad y las crisis financieras en todo el mundo, forzando a muchos países a iniciar reglas estrictas sobre el uso de mascarillas \cite{Rahmani2021}. La mejor evidencia actual incluye la posibilidad de importantes beneficios relativos y absolutos de usar una mascarilla. Esto depende de la situación de la pandemia en un determinado entorno geográfico \cite{Schunemann2020,Feng2020}.

En todo el transcurso de la pandemia se ha promovido un alto grado de cooperación científica mundial. La inteligencia artificial se destaca como una herramienta próxima y útil para identificar infecciones tempranas debidas al coronavirus y que también ayuda a monitorear el estado de los pacientes infectados. Se debe destacar su utilidad para facilitar la investigación de este virus mediante el análisis de los datos disponibles \cite{Vaishya2020,Naude2020}. Se han desarrollado diferentes técnicas, incorporando los sistemas de diagnóstico de COVID-19, como RNN, LSTM, GAN y ELM. Todas estas plataformas ayudan a los expertos en inteligencia artificial a analizar enormes conjuntos de datos y ayudar a los médicos a entrenar máquinas, establecer algoritmos u optimizar los datos analizados para tratar el virus con más velocidad y precisión \cite{Jamshidi2020}.

La detección de objetos corresponde a una técnica para la visión por computadora que está fuertemente potenciada por el Deep Learning, en donde se pueden destacar una gran cantidad de aplicaciones como autos autónomos y robots \cite{Pathak2018,Verschae2015}. El Deep Learning se puede aplicar para resolver problemas emergentes como es el caso de la detención de mascarillas, que representa una necesidad para cumplir con los protocolos de bioseguridad. Existen varios artículos que tratan modelos para la detección de mascarillas, principalmente con transferencia de aprendizaje o estructuras de redes neuronales definidas \cite{Loey2021a,Nagrath2021,UdDin2020,Loey2021}. Estos modelos poseen una gran precisión y son útiles para detectar mascarillas en amplias poblaciones, pero son considerablemente complejos y no poseen una implementación de bajo costo. Por lo tanto, se plantea un modelo sencillo de redes neuronales convolucionales utilizando la API de TensorFlow que permite una fácil transferencia a sistemas embebidos.

En el artículo se lleva a cabo un prototipo funcional con la tarjeta Raspberry Pi como eje central de operación, que supone una implementación de relativamente bajo costo para la detención de mascarillas como un sistema de control para el ingreso de distintos establecimientos de manera general, que además funciona en conjunto con un sensor de temperatura infrarrojo. Se plantea como objetivo ampliar en gran medida el conjunto de mascarillas que pueden ser detectadas correctamente, que siempre representa un inconveniente en modelos de detención de mascarillas similares.  

% Elementos usados en el producto
\section{Elementos usados en el producto}

\subsection{Raspberry Pi 3 Model B+}
La Raspberry Pi es una computadora de bajo costo del tamaño de una tarjeta de crédito que se conecta a un monitor de computadora o televisor y utiliza un teclado y un mouse estándar. La Raspberry Pi 3 Model B + es la revisión final de la gama Raspberry Pi \cite{Barkstrom2019}. En la Fig.~\ref{fig:fig1} se puede ver el modelo con sus componentes.

\begin{figure}[h]
	\centering
	    \includegraphics[scale=1.3]{1. Raspberry Pi 3}
    \caption{Raspberry Pi 3 Model B+ \cite{Raspberry2016a}. }
    \label{fig:fig1} 
\end{figure}

\begin{figure*}[b]
	\centering
	    \includegraphics[scale=0.7]{3. Convolutional layer}
    \caption{Uso de la red neuronal convolucional para la detección de objetos\cite{Pathak2018}. }
    \label{fig:fig2} 
\end{figure*}

\subsection{Raspberry Pi Camera Module V1}
El módulo de cámara Raspberry Pi se puede utilizar para tomar videos de alta definición, así como también fotografías. Posee una resolución de 5 Megapíxeles. La cámara funciona con todos los modelos de Raspberry Pi 1, 2, 3 y 4. Se puede acceder a ella a través de las API MMAL y V4L, y existen numerosas bibliotecas de terceros creadas para ella, incluida la biblioteca Picamera Python \cite{Raspberry2016b,Raspberry2018}.

\subsection{MLX90614 Gy-906}
Es un sensor infrarrojo diseñado para medir temperatura sin contacto, ese sensor tiene un conversor de 17 bits interno y un DSP que procesa los datos para dar una alta resolución. El termómetro viene calibrado de fábrica. El rango de temperatura es de -20 \textcelsius~a 120 \textcelsius, con una resolución de salida de 0.14 \textcelsius \cite{MLX2006}.

\subsection{Colaboratory}
Colab es un servicio cloud, basado en los Notebooks de Jupyter, que permite escribir y ejecutar código de Python en un navegador sin configuración requerida, con acceso gratuito a GPUs y TPUs de Google, cotiene librerías como: Scikit-learn, PyTorch, TensorFlow, Keras y OpenCV. Todo ello con bajo Python 2.7 y 3.7 \cite{Colaboratory}.

\subsection{TensorFlow}
TensorFlow es una plataforma de código abierto de extremo a extremo para el aprendizaje automático. Cuenta con un ecosistema integral y flexible de herramientas, bibliotecas y recursos de la comunidad que les permite a los investigadores impulsar un aprendizaje automático innovador y, a los desarrolladores, compilar e implementar con facilidad aplicaciones con tecnología de AA \cite{Google,Google2015}.

% Metodología y desarrollo 
\section{Metodología y desarrollo}
Para predecir si una persona está portando una mascarilla, inicialmente se debe entrenar un modelo de Deep Learning usando un conjunto de datos adecuado. Los detalles sobre el conjunto de datos se discuten en la sección \ref{sec:sec1}. Para entrenar el clasificador es necesario primero realizar una limpieza y preprocesamiento de datos, después los datos se organizan en un conjunto de entrenamiento y de validación, estos datos se separan por lotes definidos para mejorar el entrenamiento del modelo. Una vez entrenado el modelo, se evalúa el rendimiento basado en su precisión y perdida, los aspectos sobre los métodos utilizados se profundizarán con más detalle en la sección \ref{sec:sec4}.

Con el modelo entrenado y optimizado, es necesario una transformación a un modelo de TFLite, de este modo es posible una fácil ejecución en sistemas embebidos o con pocos recursos, para este caso práctico, a la Raspberry Pi. La Fig.~\ref{fig:fig2} presenta un diagrama de flujo con todo el proceso desarrollado para el modelo.



\subsection{Conjunto de datos}
\label{sec:sec1}
El modelo se entrena con dos conjuntos de datos para evaluar el rendimiento y velocidad de predicción. El primer conjunto de datos es generado artificialmente por Prajna Bhandary, en donde toma imágenes estándar de rostros y aplica puntos de referencia faciales. Los puntos de referencia faciales permitían localizar los rasgos faciales de una persona como ojos, cejas, nariz, boca y mandíbula. Esto permite crear un conjunto de datos artificialmente al incluir una mascarilla en la imagen de una persona sin mascarilla. Este conjunto de datos cuenta con 1376 imágenes, con la etiqueta "with\_mask" y "without\_mask", se encuentra dispoible en: \url{https://github.com/prajnasb/observations.git}. 

El segundo conjunto de datos corresponde a una combinación de varios conjuntos de datos e imágenes de código abierto que contiene una gran variedad de etnias y nacionalidades.

\subsection{Data augmentation}
\label{sec:sec2}
Para un buen desempeño del modelo, es necesario una gran cantidad de datos para realizar el entrenamiento de manera efectiva, esto debido a que no se cuenta con la disponibilidad de una cantidad adecuada de datos para entrenar el modelo propuesto. El método de aumento de datos se utiliza para resolver este problema. En esta técnica, se utilizan métodos como la rotación, el zoom, el desplazamiento, la distorsión y la inversión de la imagen para aumentar artificialmente la diversidad de conjuntos de datos para entrenar detectores. Para esto se utiliza una función de generación de datos de imagen para el aumento de imágenes, que devuelve lotes de datos de prueba y entrenamiento.
\subsection{Arquitectura de la red neuronal}
\label{sec:sec3}
\subsection{Método de optimización y función de perdida}
\label{sec:sec4}
\subsection{Hiperparámetros}
\label{sec:sec5}    
\subsection{Entrenamiento de la red neuronal}
\label{sec:sec6}
\subsection{Sensor infrarrojo}
\label{sec:sec7}
\subsection{Algoritmo en Python}
\label{sec:sec8}

% Resultados
\section{Resultados}

% Limitaciones
\section{Limitaciones}

% Conclusiones y futuros trabajos
\section{Conclusiones y futuros trabajos}


% Reefrencias
\bibliographystyle{IEEEtran}
\bibliography{library}

\end{document}


